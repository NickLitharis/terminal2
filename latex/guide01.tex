\textbf{DISCLAIMER}: Διαβάστε ολόκληρες τις οδηγίες μέχρι το τέλος πριν
ξεκινήσετε. Οι οδηγίες καλό είναι ναι μεν να ακολουθούνται κατά γράμμα
αλλά όχι δογματικά, μην τις εκτελείτε στα τυφλά, διαβάστε και
αναρωτηθείτε τι κάνει η κάθε εντολή και γιατί την εκτελούμε. Μόνο έτσι
θα καταλάβετε πραγματικά τη λογική των παραδοτέων και των εργαλείων που
χρησιμοποιούμε.

\hypertarget{ux3c0ux3c1ux3bfux3b1ux3c0ux3b1ux3b9ux3c4ux3bfux3cdux3bcux3b5ux3bdux3b1}{%
\section{Προαπαιτούμενα}\label{ux3c0ux3c1ux3bfux3b1ux3c0ux3b1ux3b9ux3c4ux3bfux3cdux3bcux3b5ux3bdux3b1}}

\begin{enumerate}
\def\labelenumi{\arabic{enumi}.}
\tightlist
\item
  Έχουμε εγκατεστημένη μία Linux διανομή στο σύστημά μας και κατά
  προτίμηση τα Arch.
\item
  Έχουμε κάνει fork στο προφίλ μας το \textbf{site} από τον οργανισμό
  μας.
\item
  Έχουμε κάνει fork στο προφίλ μας τα \textbf{submodules} από τον
  οργανισμό μας.
\end{enumerate}

\hypertarget{clone-ux3c4ux3bfux3c5-ux3b1ux3c0ux3bfux3b8ux3b5ux3c4ux3b7ux3c1ux3afux3bfux3c5-ux3c3ux3c4ux3bf-ux3bcux3b7ux3c7ux3acux3bdux3b7ux3bcux3ac-ux3bcux3b1ux3c2}{%
\section{Clone του αποθετηρίου στο μηχάνημά
μας}\label{clone-ux3c4ux3bfux3c5-ux3b1ux3c0ux3bfux3b8ux3b5ux3c4ux3b7ux3c1ux3afux3bfux3c5-ux3c3ux3c4ux3bf-ux3bcux3b7ux3c7ux3acux3bdux3b7ux3bcux3ac-ux3bcux3b1ux3c2}}

\begin{enumerate}
\def\labelenumi{\arabic{enumi}.}
\tightlist
\item
  \texttt{git\ clone\ https://github.com/*your-username*/site}
\end{enumerate}

\hypertarget{ux3b5ux3c0ux3b5ux3beux3b5ux3c1ux3b3ux3b1ux3c3ux3afux3b1}{%
\section{Επεξεργασία}\label{ux3b5ux3c0ux3b5ux3beux3b5ux3c1ux3b3ux3b1ux3c3ux3afux3b1}}

\hypertarget{ux3c4ux3c1ux3ccux3c0ux3bfux3c2-1}{%
\subsubsection{Τρόπος 1}\label{ux3c4ux3c1ux3ccux3c0ux3bfux3c2-1}}

\begin{enumerate}
\def\labelenumi{\arabic{enumi}.}
\tightlist
\item
  \texttt{true\ \textgreater{}\ .gitmodules} (Αδειάζει το αρχείο)
\item
  \texttt{git\ add\ .}
\item
  \texttt{git\ rm\ -\/-cached\ \_gallery\ \_bibliography\ \_notes\ \_quotes\ images}
\item
  \texttt{rm\ -R\ \_gallery\ \_bibliography\ \_notes\ \_quotes\ images}
\item
  \texttt{cd\ \_includes}
\item
  \texttt{git\ rm\ -\/-cached\ extras\ text}
\item
  \texttt{rm\ -R\ extras\ text}
\item
  \texttt{git\ add\ .}
\item
  \texttt{git\ commit\ -m\ "your\ message"}
\item
  \texttt{git\ push\ origin}
\item
  \texttt{cd\ ..}
\end{enumerate}

\hypertarget{ux3c0ux3c1ux3bfux3c3ux3b8ux3aeux3baux3b7-ux3c4ux3c9ux3bd-submodules}{%
\paragraph{Προσθήκη των
submodules}\label{ux3c0ux3c1ux3bfux3c3ux3b8ux3aeux3baux3b7-ux3c4ux3c9ux3bd-submodules}}

\begin{enumerate}
\def\labelenumi{\arabic{enumi}.}
\tightlist
\item
  \texttt{git\ submodule\ add\ https://github.com/*your-username*/\_gallery}
\item
  \texttt{git\ submodule\ add\ https://github.com/*your-username*/\_quotes}
\item
  \texttt{git\ submodule\ add\ https://github.com/*your-username*/images}
\item
  \texttt{git\ submodule\ add\ https://github.com/*your-username*/bibliography\ \_bibliography}
\item
  \texttt{git\ submodule\ add\ https://github.com/*your-username*/notes\ \_notes}
\item
  \texttt{cd\ \_includes}
\item
  \texttt{git\ submodule\ add\ https://github.com/*your-username*/extras}
\item
  \texttt{git\ submodule\ add\ https://github.com/*your-username*/text}
\item
  \texttt{cd\ ..}
\item
  \texttt{git\ add\ .}
\item
  \texttt{git\ commit\ -m\ "your-message"}
\item
  \texttt{git\ push\ origin}
\end{enumerate}

\hypertarget{ux3c4ux3c1ux3ccux3c0ux3bfux3c2-2}{%
\subsubsection{Τρόπος 2}\label{ux3c4ux3c1ux3ccux3c0ux3bfux3c2-2}}

\begin{enumerate}
\def\labelenumi{\arabic{enumi}.}
\tightlist
\item
  αλλαγή στα link των submodules όπως στο βήμα \textbf{\emph{Προσθήκη
  των submodules}} στο αρχείο .gitmodules
\item
  \texttt{git\ submodule\ update\ -\/-remote\ -\/-init}
\item
  \texttt{git\ submodule\ update\ -\/-remote\ -\/-merge}
\end{enumerate}

\textbf{Πριν κάνετε τις δικές σας αλλαγές πάτε στην ενότητα Netlify για
να βεβαιωθείτε ότι το site λειτουργεί σωστά}

\hypertarget{ux3b5ux3b9ux3c3ux3b1ux3b3ux3c9ux3b3ux3ae-ux3c4ux3c9ux3bd-ux3b1ux3c1ux3c7ux3b5ux3afux3c9ux3bd-ux3bcux3b1ux3c2}{%
\section{Εισαγωγή των αρχείων
μας}\label{ux3b5ux3b9ux3c3ux3b1ux3b3ux3c9ux3b3ux3ae-ux3c4ux3c9ux3bd-ux3b1ux3c1ux3c7ux3b5ux3afux3c9ux3bd-ux3bcux3b1ux3c2}}

Τα \href{https://courses-ionio.github.io/help/social/}{αρχεία}.

\begin{enumerate}
\def\labelenumi{\arabic{enumi}.}
\tightlist
\item
  Τοποθετούμε τα αρχεία \texttt{.md} στο φάκελο \texttt{\_gallery}
\item
  Τοποθετούμε τις εικόνες στο φάκελο \texttt{images}
\end{enumerate}

\hypertarget{ux3b1ux3bdux3adux3b2ux3b1ux3c3ux3bcux3b1-ux3c4ux3c9ux3bd-ux3b1ux3c1ux3c7ux3b5ux3afux3c9ux3bd}{%
\section{Ανέβασμα των
αρχείων}\label{ux3b1ux3bdux3adux3b2ux3b1ux3c3ux3bcux3b1-ux3c4ux3c9ux3bd-ux3b1ux3c1ux3c7ux3b5ux3afux3c9ux3bd}}

\begin{enumerate}
\def\labelenumi{\arabic{enumi}.}
\tightlist
\item
  Βλέπουμε τι αλλαγές έχουμε κάνει και σε ποιό φάκελο με την εντολή
  \texttt{git\ status}
\item
  Κάνουμε \texttt{cd} στους φακέλους που έχουμε κάνει τις αλλαγές και:
\end{enumerate}

\begin{itemize}
\tightlist
\item
  \texttt{git\ add\ .}
\item
  \texttt{git\ commit\ -m\ "your-message"}
\item
  \texttt{git\ push\ origin}
\end{itemize}

\begin{enumerate}
\def\labelenumi{\arabic{enumi}.}
\setcounter{enumi}{14}
\tightlist
\item
  Κάνουμε το ίδιο και στο directory \textbf{\emph{site}} και ελέγχουμε
  με το \texttt{git\ status} μέχρι να μας βγάλει το μήνυμα:
\end{enumerate}

\begin{verbatim}
On branch master
Your branch is up to date with 'origin/master'.
nothing to commit, working tree clean
\end{verbatim}

\hypertarget{netlify}{%
\section{Netlify}\label{netlify}}

\begin{enumerate}
\def\labelenumi{\arabic{enumi}.}
\tightlist
\item
  Συνδεόμαστε στην πλατφόρμα του Netlify μέσω του λογαριασμού μας στο
  GitHub
\item
  Επιλέγουμε \texttt{Add\ a\ new\ site} →
  \texttt{Import\ an\ existing\ project}
\item
  Επιλέγουμε το GitHub και δίνουμε τις απαραίτητες άδειες
\item
  Πατάμε το \texttt{Only\ select\ repositories} και επιλέγουμε το
  \texttt{site}
\item
  Επιλέγουμε το μενού \texttt{Show\ advanced} → \texttt{New\ variable}
  και δίνουμε ως key το \texttt{RUBY\_VERSION} και ως value το
  \texttt{2.6.2.47}
\item
  Κάνουμε deploy το site
\end{enumerate}

Θα πρέπει να μπορούμε να δούμε το περιεχόμενο που προσθέσαμε

\hypertarget{pull-request}{%
\section{Pull request}\label{pull-request}}

\begin{enumerate}
\def\labelenumi{\arabic{enumi}.}
\tightlist
\item
  Πηγαίνουμε στα αποθετήρια μας που έχουμε κάνει αλλαγές και πατάμε το
  κουμπί \texttt{contribute} → \texttt{create\ a\ pull\ request}
\item
  Γράφουμε μια σύντομη περιγραφή των αλλαγών που έχουμε κάνει και
  δίνουμε τα links από την ιστοσελίδα μας που πιστοποιούν ότι οι αλλαγές
  μας δουλεύουν και περιμένουμε την έγκριση από τους διαχειριστές
\end{enumerate}

\hypertarget{ux3c0ux3b1ux3c1ux3b1ux3c4ux3b7ux3c1ux3aeux3c3ux3b5ux3b9ux3c2}{%
\section{Παρατηρήσεις}\label{ux3c0ux3b1ux3c1ux3b1ux3c4ux3b7ux3c1ux3aeux3c3ux3b5ux3b9ux3c2}}

\begin{enumerate}
\def\labelenumi{\arabic{enumi}.}
\tightlist
\item
  Για τα pull request επιλέγουμε μόνο τα submodules πού έχουμε κάνει
  αλλαγές και όχι το site
\item
  \textbf{Tip:} Είναι καλύτερο να κάνουμε \texttt{add} και
  \texttt{commit} συγκεκριμένα files κάθε φορά και να δίνουμε ακριβή
  περιγραφή των αλλαγών που έχουμε κάνει. Με αυτό τον τρόπο είναι πιο
  εύκολο να ελεγχθούν τα αρχεία σε ένα pull request.
\item
  \textbf{Tip:} Σε περίπτωση που κάποιος άλλος από την ομάδα μας έχει
  κάνει επιτυχώς ένα pull request πριν από εμάς θα πρέπει να
  ακολουθήσουμε δύο επιπλέον βήματα:
\end{enumerate}

\begin{itemize}
\tightlist
\item
  Βλέπουμε αν τα forked αποθετήρια μας είναι synced με το αποθετήριο του
  οργανισμού μας, αν όχι τότε πατάμε το κουμπί sync. Καλό είναι να
  κάνουμε sync πριν ανεβάσουμε εμείς τις δικές μας αλλαγές για να
  αποφεύγονται συγκρούσεις.
\item
  Στο terminal χρησιμοποιούμε την εντολή \texttt{git\ pull\ origin} αφού
  έχουμε δει ποιοι φάκελοι χρειάζονται αλλαγές με την εντολή
  \texttt{git\ status}
\end{itemize}
