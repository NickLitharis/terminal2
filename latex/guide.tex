% Options for packages loaded elsewhere
\PassOptionsToPackage{unicode}{hyperref}
\PassOptionsToPackage{hyphens}{url}
%
\documentclass[
]{article}
\usepackage{amsmath,amssymb}
\usepackage{lmodern}
\usepackage{iftex}
\ifPDFTeX
  \usepackage[T1]{fontenc}
  \usepackage[utf8]{inputenc}
  \usepackage{textcomp} % provide euro and other symbols
\else % if luatex or xetex
  \usepackage{unicode-math}
  \defaultfontfeatures{Scale=MatchLowercase}
  \defaultfontfeatures[\rmfamily]{Ligatures=TeX,Scale=1}
\fi
% Use upquote if available, for straight quotes in verbatim environments
\IfFileExists{upquote.sty}{\usepackage{upquote}}{}
\IfFileExists{microtype.sty}{% use microtype if available
  \usepackage[]{microtype}
  \UseMicrotypeSet[protrusion]{basicmath} % disable protrusion for tt fonts
}{}
\makeatletter
\@ifundefined{KOMAClassName}{% if non-KOMA class
  \IfFileExists{parskip.sty}{%
    \usepackage{parskip}
  }{% else
    \setlength{\parindent}{0pt}
    \setlength{\parskip}{6pt plus 2pt minus 1pt}}
}{% if KOMA class
  \KOMAoptions{parskip=half}}
\makeatother
\usepackage{xcolor}
\usepackage{graphicx}
\makeatletter
\def\maxwidth{\ifdim\Gin@nat@width>\linewidth\linewidth\else\Gin@nat@width\fi}
\def\maxheight{\ifdim\Gin@nat@height>\textheight\textheight\else\Gin@nat@height\fi}
\makeatother
% Scale images if necessary, so that they will not overflow the page
% margins by default, and it is still possible to overwrite the defaults
% using explicit options in \includegraphics[width, height, ...]{}
\setkeys{Gin}{width=\maxwidth,height=\maxheight,keepaspectratio}
% Set default figure placement to htbp
\makeatletter
\def\fps@figure{htbp}
\makeatother
\setlength{\emergencystretch}{3em} % prevent overfull lines
\providecommand{\tightlist}{%
  \setlength{\itemsep}{0pt}\setlength{\parskip}{0pt}}
\setcounter{secnumdepth}{-\maxdimen} % remove section numbering
\ifLuaTeX
  \usepackage{selnolig}  % disable illegal ligatures
\fi
\IfFileExists{bookmark.sty}{\usepackage{bookmark}}{\usepackage{hyperref}}
\IfFileExists{xurl.sty}{\usepackage{xurl}}{} % add URL line breaks if available
\urlstyle{same} % disable monospaced font for URLs
\hypersetup{
  hidelinks,
  pdfcreator={LaTeX via pandoc}}

\author{}
\date{}

\begin{document}

\textbf{DISCLAIMER}: Διαβάστε ολόκληρες τις οδηγίες μέχρι το τέλος πριν
ξεκινήσετε. Οι οδηγίες καλό είναι ναι μεν να ακολουθούνται κατά γράμμα
αλλά όχι δογματικά, μην τις εκτελείτε στα τυφλά, διαβάστε και
αναρωτηθείτε τι κάνει η κάθε εντολή και γιατί την εκτελούμε. Μόνο έτσι
θα καταλάβετε πραγματικά τη λογική των παραδοτέων και των εργαλείων που
χρησιμοποιούμε.

\hypertarget{ux3c0ux3c1ux3bfux3b1ux3c0ux3b1ux3b9ux3c4ux3bfux3cdux3bcux3b5ux3bdux3b1}{%
\section{Προαπαιτούμενα}\label{ux3c0ux3c1ux3bfux3b1ux3c0ux3b1ux3b9ux3c4ux3bfux3cdux3bcux3b5ux3bdux3b1}}

\begin{enumerate}
\item
  Έχουμε εγκατεστημένη μία Linux διανομή στο σύστημά μας και κατά
  προτίμηση τα Arch.
\item
  Έχουμε κάνει fork στο προφίλ μας το \textbf{site} από τον οργανισμό
  μας.
\item
  Έχουμε κάνει fork στο προφίλ μας τα \textbf{submodules} από τον
  οργανισμό μας.
\end{enumerate}

\hypertarget{clone-ux3c4ux3bfux3c5-ux3b1ux3c0ux3bfux3b8ux3b5ux3c4ux3b7ux3c1ux3afux3bfux3c5-ux3c3ux3c4ux3bf-ux3bcux3b7ux3c7ux3acux3bdux3b7ux3bcux3ac-ux3bcux3b1ux3c2}{%
\section{Clone του αποθετηρίου στο μηχάνημά
μας}\label{clone-ux3c4ux3bfux3c5-ux3b1ux3c0ux3bfux3b8ux3b5ux3c4ux3b7ux3c1ux3afux3bfux3c5-ux3c3ux3c4ux3bf-ux3bcux3b7ux3c7ux3acux3bdux3b7ux3bcux3ac-ux3bcux3b1ux3c2}}

\begin{enumerate}
\item
  \texttt{git~clone~https://github.com/*your-username*/site}
\end{enumerate}

\hypertarget{ux3b5ux3c0ux3b5ux3beux3b5ux3c1ux3b3ux3b1ux3c3ux3afux3b1}{%
\section{Επεξεργασία}\label{ux3b5ux3c0ux3b5ux3beux3b5ux3c1ux3b3ux3b1ux3c3ux3afux3b1}}

\hypertarget{ux3c4ux3c1ux3ccux3c0ux3bfux3c2-1}{%
\subsubsection{Τρόπος 1}\label{ux3c4ux3c1ux3ccux3c0ux3bfux3c2-1}}

\begin{enumerate}
\item
  \texttt{true~\textgreater{}~.gitmodules} (Αδειάζει το αρχείο)
\item
  \texttt{git~add~.}
\item
  \texttt{git~rm~-\/-cached~\_gallery~\_bibliography~\_notes~\_quotes~images}
\item
  \texttt{rm~-R~\_gallery~\_bibliography~\_notes~\_quotes~images}
\item
  \texttt{cd~\_includes}
\item
  \texttt{git~rm~-\/-cached~extras~text}
\item
  \texttt{rm~-R~extras~text}
\item
  \texttt{git~add~.}
\item
  \texttt{git~commit~-m~"your~message"}
\item
  \texttt{git~push~origin}
\item
  \texttt{cd~..}
\end{enumerate}

\hypertarget{ux3c0ux3c1ux3bfux3c3ux3b8ux3aeux3baux3b7-ux3c4ux3c9ux3bd-submodules}{%
\paragraph{Προσθήκη των
submodules}\label{ux3c0ux3c1ux3bfux3c3ux3b8ux3aeux3baux3b7-ux3c4ux3c9ux3bd-submodules}}

\begin{enumerate}
\item
  \texttt{git~submodule~add~https://github.com/*your-username*/\_gallery}
\item
  \texttt{git~submodule~add~https://github.com/*your-username*/\_quotes}
\item
  \texttt{git~submodule~add~https://github.com/*your-username*/images}
\item
  \texttt{git~submodule~add~https://github.com/*your-username*/bibliography~\_bibliography}
\item
  \texttt{git~submodule~add~https://github.com/*your-username*/notes~\_notes}
\item
  \texttt{cd~\_includes}
\item
  \texttt{git~submodule~add~https://github.com/*your-username*/extras}
\item
  \texttt{git~submodule~add~https://github.com/*your-username*/text}
\item
  \texttt{cd~..}
\item
  \texttt{git~add~.}
\item
  \texttt{git~commit~-m~"your-message"}
\item
  \texttt{git~push~origin}
\end{enumerate}

\hypertarget{ux3c4ux3c1ux3ccux3c0ux3bfux3c2-2}{%
\subsubsection{Τρόπος 2}\label{ux3c4ux3c1ux3ccux3c0ux3bfux3c2-2}}

\begin{enumerate}
\item
  αλλαγή στα link των submodules όπως στο βήμα \textbf{\emph{Προσθήκη
  των submodules}} στο αρχείο .gitmodules
\item
  \texttt{git~submodule~update~-\/-remote~-\/-init}
\item
  \texttt{git~submodule~update~-\/-remote~-\/-merge}
\end{enumerate}

\textbf{Πριν κάνετε τις δικές σας αλλαγές πάτε στην ενότητα Netlify για
να βεβαιωθείτε ότι το site λειτουργεί σωστά}

\hypertarget{ux3b5ux3b9ux3c3ux3b1ux3b3ux3c9ux3b3ux3ae-ux3c4ux3c9ux3bd-ux3b1ux3c1ux3c7ux3b5ux3afux3c9ux3bd-ux3bcux3b1ux3c2}{%
\section{Εισαγωγή των αρχείων
μας}\label{ux3b5ux3b9ux3c3ux3b1ux3b3ux3c9ux3b3ux3ae-ux3c4ux3c9ux3bd-ux3b1ux3c1ux3c7ux3b5ux3afux3c9ux3bd-ux3bcux3b1ux3c2}}

Τα \href{https://courses-ionio.github.io/help/social/}{αρχεία}.

\begin{enumerate}
\item
  Τοποθετούμε τα αρχεία \texttt{.md} στο φάκελο \texttt{\_gallery}
\item
  Τοποθετούμε τις εικόνες στο φάκελο \texttt{images}
\end{enumerate}

\hypertarget{ux3b1ux3bdux3adux3b2ux3b1ux3c3ux3bcux3b1-ux3c4ux3c9ux3bd-ux3b1ux3c1ux3c7ux3b5ux3afux3c9ux3bd}{%
\section{Ανέβασμα των
αρχείων}\label{ux3b1ux3bdux3adux3b2ux3b1ux3c3ux3bcux3b1-ux3c4ux3c9ux3bd-ux3b1ux3c1ux3c7ux3b5ux3afux3c9ux3bd}}

\begin{enumerate}
\item
  Βλέπουμε τι αλλαγές έχουμε κάνει και σε ποιό φάκελο με την εντολή
  \texttt{git~status}
\item
  Κάνουμε \texttt{cd} στους φακέλους που έχουμε κάνει τις αλλαγές και:
\end{enumerate}

\begin{itemize}
\item
  \texttt{git~add~.}
\item
  \texttt{git~commit~-m~"your-message"}
\item
  \texttt{git~push~origin}
\end{itemize}

\begin{enumerate}
\item
  Κάνουμε το ίδιο και στο directory \textbf{\emph{site}} και ελέγχουμε
  με το \texttt{git~status} μέχρι να μας βγάλει το μήνυμα:
\end{enumerate}

\begin{verbatim}
On branch master
Your branch is up to date with 'origin/master'.
nothing to commit, working tree clean
\end{verbatim}

\hypertarget{netlify}{%
\section{Netlify}\label{netlify}}

\begin{enumerate}
\item
  Συνδεόμαστε στην πλατφόρμα του Netlify μέσω του λογαριασμού μας στο
  GitHub
\item
  Επιλέγουμε \texttt{Add~a~new~site} →
  \texttt{Import~an~existing~project}
\item
  Επιλέγουμε το GitHub και δίνουμε τις απαραίτητες άδειες
\item
  Πατάμε το \texttt{Only~select~repositories} και επιλέγουμε το
  \texttt{site}
\item
  Επιλέγουμε το μενού \texttt{Show~advanced} → \texttt{New~variable} και
  δίνουμε ως key το \texttt{RUBY\_VERSION} και ως value το
  \texttt{2.6.2.47}
\item
  Κάνουμε deploy το site
\end{enumerate}

Θα πρέπει να μπορούμε να δούμε το περιεχόμενο που προσθέσαμε

\hypertarget{pull-request}{%
\section{Pull request}\label{pull-request}}

\begin{enumerate}
\item
  Πηγαίνουμε στα αποθετήρια μας που έχουμε κάνει αλλαγές και πατάμε το
  κουμπί \texttt{contribute} → \texttt{create~a~pull~request}
\item
  Γράφουμε μια σύντομη περιγραφή των αλλαγών που έχουμε κάνει και
  δίνουμε τα links από την ιστοσελίδα μας που πιστοποιούν ότι οι αλλαγές
  μας δουλεύουν και περιμένουμε την έγκριση από τους διαχειριστές
\end{enumerate}

\hypertarget{ux3c0ux3b1ux3c1ux3b1ux3c4ux3b7ux3c1ux3aeux3c3ux3b5ux3b9ux3c2}{%
\section{Παρατηρήσεις}\label{ux3c0ux3b1ux3c1ux3b1ux3c4ux3b7ux3c1ux3aeux3c3ux3b5ux3b9ux3c2}}

\begin{enumerate}
\item
  Για τα pull request επιλέγουμε μόνο τα submodules πού έχουμε κάνει
  αλλαγές και όχι το site
\item
  \textbf{Tip:} Είναι καλύτερο να κάνουμε \texttt{add} και
  \texttt{commit} συγκεκριμένα files κάθε φορά και να δίνουμε ακριβή
  περιγραφή των αλλαγών που έχουμε κάνει. Με αυτό τον τρόπο είναι πιο
  εύκολο να ελεγχθούν τα αρχεία σε ένα pull request.
\item
  \textbf{Tip:} Σε περίπτωση που κάποιος άλλος από την ομάδα μας έχει
  κάνει επιτυχώς ένα pull request πριν από εμάς θα πρέπει να
  ακολουθήσουμε δύο επιπλέον βήματα:
\end{enumerate}

\begin{itemize}
\item
  Βλέπουμε αν τα forked αποθετήρια μας είναι synced με το αποθετήριο του
  οργανισμού μας, αν όχι τότε πατάμε το κουμπί sync. Καλό είναι να
  κάνουμε sync πριν ανεβάσουμε εμείς τις δικές μας αλλαγές για να
  αποφεύγονται συγκρούσεις.
\item
  Στο terminal χρησιμοποιούμε την εντολή \texttt{git~pull~origin} αφού
  έχουμε δει ποιοι φάκελοι χρειάζονται αλλαγές με την εντολή
  \texttt{git~status}
\end{itemize}

\hypertarget{netlify-cli}{%
\section{Netlify-CLI}\label{netlify-cli}}

\hypertarget{windows}{%
\subsection{Windows}\label{windows}}

Απαραίτητα: \href{https://nodejs.org/en/download/}{node.js},
\href{https://rubyinstaller.org/downloads/}{ruby + jekyll},
\href{https://cli.netlify.com/}{Netlify cli} \#\#\# node.js install

\begin{itemize}
\item
  Ακολουθούμε απλά το install του και για καλό και για κακό βάζουμε και
  την επιλογή να κάνει install και το Chocolatey (δε δοκιμάστηκε χωρίς
  αυτό)
\end{itemize}

\hypertarget{ruby-jekyll}{%
\subsubsection{ruby + jekyll}\label{ruby-jekyll}}

\begin{itemize}
\item
  Ανοίγουμε το installer
\end{itemize}

\begin{figure}
\centering
\includegraphics{https://user-images.githubusercontent.com/45509916/200146200-ee3e7b04-9d70-41c3-9783-35678d46a9c2.png}
\caption{image}
\end{figure}

Σε αυτή τη φάση κάνουμε και τις 3 επιλογές

\begin{figure}
\centering
\includegraphics{https://user-images.githubusercontent.com/45509916/200145993-08c1436b-ce9c-4684-a170-1c82474a449c.png}
\caption{image}
\end{figure}

\begin{itemize}
\item
  Μόλις τελειώσει με τις επιλογές μπορούμε άπλα να το κλείσουμε.
\item
  Κοιτάζουμε αν το έχουμε εγκαταστήσει σωστά με την εντολή
  \texttt{jekyll~-v}
\item
  Στη συνέχεια χρησιμοποιούμε την εντολή \texttt{bundler~install} για να
  μας δημιουργήσει όλα τα απαραίτητα gems που χρειαζόμαστε.
\item
  Μετά από αυτό κάλο είναι να γίνει ένα reboot (το συνιστά και το
  installer)
\end{itemize}

\hypertarget{netlify-cli-setup}{%
\subsubsection{Netlify cli setup}\label{netlify-cli-setup}}

\begin{itemize}
\item
  Αφού έχουμε τελειώσει με όλα τα install
\item
  χρησιμοποιούμε την εντολή \texttt{npm~install~netlify-cli~-g} για την
  εγκατάσταση του Netlify cli
\item
  κάνουμε \texttt{netlify} για να δούμε ότι έχει γίνει η εγκατάσταση
  όπου θα μας εμφανίσει το παρακάτω πίνακα
\end{itemize}

\begin{figure}
\centering
\includegraphics{https://user-images.githubusercontent.com/45509916/200147017-2995fe4f-2c45-4e80-a03b-960d2ffddd0a.png}
\caption{image}
\end{figure}

\textbf{ΔΕΝ ΞΕΧΝΑΜΕ ΝΑ ΒΑΛΟΥΜΕ ΤΗ ΣΩΣΤΗ VERSION ΤΗΣ RUBY netlify env:set
RUBY\_VERSION 2.6.2.47}

\begin{enumerate}
\item
  Χρησιμοποιούμε την εντολή \texttt{netlify~login} όπου θα μας ανοίξει
  τη σελίδα του Netlify για να κάνουμε authorize το τερματικό μας, στη
  συνέχεια πατάμε \texttt{netlify~init}
\item
  πατάμε την εντολή που μας λέει να συνδεθούμε σε ήδη υπάρχον site και
  στη συνέχεια θα πατήσουμε για το repository που είναι το site μας.
\item
  \texttt{netlify~build} για να χτιστεί η σελίδα μας και
  \texttt{netlify~deploy} για να γίνει deploy η σελίδα σε ένα draft link
  στο οποίο μπορείτε να τεστάρετε τις αλλαγές σας χωρίς να χρειάζεστε να
  κάνετε commit πολλά πέρα από αυτά που χρειάζεστε και να έχετε ένα
  πολύπλοκο history στο repository σας.
\end{enumerate}

\hypertarget{arch-linux}{%
\subsection{Arch Linux}\label{arch-linux}}

\begin{enumerate}
\item
  Εγκαθιστούμε τα nodejs, ruby
\end{enumerate}

\begin{itemize}
\item
  \texttt{sudo~pacman~-S~nodejs}
\item
  \texttt{sudo~pacman~-S~ruby~base-devel}
\end{itemize}

\begin{enumerate}
\item
  Προσθέτουμε το ruby στο PATH (το ίδιο ισχύει για όλα τα shells)
\end{enumerate}

\begin{itemize}
\item
  \texttt{echo~\#~Install~Ruby~Gems~to~\textasciitilde{}/gems~\textgreater{}\textgreater{}~\textasciitilde{}/.bashrc}
  *
\item
  \texttt{echo~export~GEM\_HOME="\$HOME/gems"~\textgreater{}\textgreater{}~\textasciitilde{}/.bashrc}
\item
  \texttt{echo~export~PATH="\$HOME/gems/bin:\$PATH"~\textgreater{}\textgreater{}~\textasciitilde{}/.bashrc}
\item
  \texttt{source~\textasciitilde{}/.bashrc}
\end{itemize}

\begin{enumerate}
\item
  Εγκαθιστούμε το Jekyll και το Bundler
\end{enumerate}

\begin{itemize}
\item
  \texttt{gem~install~jekyll~bundler}
\item
  \texttt{bundler~install}
\end{itemize}

\begin{enumerate}
\item
  Εγκαθιστούμε το Netlify
\end{enumerate}

\begin{itemize}
\item
  \texttt{npm~install~netlify-cli~-g}
\end{itemize}

Ακολουθούμε τις οδηγίες που αναφέρονται για τα Windows

\hypertarget{ux3baux3b1ux3c4ux3b1ux3c3ux3baux3b5ux3c5ux3ae-ux3b2ux3b9ux3b2ux3bbux3afux3bfux3c5}{%
\section{Κατασκευή
Βιβλίου}\label{ux3baux3b1ux3c4ux3b1ux3c3ux3baux3b5ux3c5ux3ae-ux3b2ux3b9ux3b2ux3bbux3afux3bfux3c5}}

\hypertarget{ux3c0ux3c1ux3bfux3b1ux3c0ux3b1ux3b9ux3c4ux3bfux3cdux3bcux3b5ux3bdux3b1}{%
\subsection{Προαπαιτούμενα}\label{ux3c0ux3c1ux3bfux3b1ux3c0ux3b1ux3b9ux3c4ux3bfux3cdux3bcux3b5ux3bdux3b1}}

\begin{enumerate}
\item
  \texttt{sudo~pacman~-S~pandoc}
\item
  \texttt{sudo~pacman~-S~texlive-most} (2GB install)
\item
  \texttt{sudo~pip~install~pandoc-fignos}
\end{enumerate}

\hypertarget{ux3b5ux3b3ux3baux3b1ux3c4ux3acux3c3ux3c4ux3b1ux3c3ux3b7-fonts}{%
\subsubsection{Εγκατάσταση
fonts}\label{ux3b5ux3b3ux3baux3b1ux3c4ux3acux3c3ux3c4ux3b1ux3c3ux3b7-fonts}}

\begin{enumerate}
\item
  Βρισκόμαστε στο \texttt{\textasciitilde{}/} directory
\item
  \texttt{ls~-la}
\item
  Βλέπουμε αν υπάρχει ο φάκελος \texttt{.fonts} αν δεν υπάρχει τότε
  \texttt{mkdir~.fonts}
\item
  κατεβάζουμε σε zip το font που θέλουμε από την ιστοσελίδα του
\item
  unzip font-που-κατεβάσαμε.zip -d \textasciitilde/.fonts
\end{enumerate}

optional: \texttt{yay~community/ttf-meslo-nerd-font-powerlevel10k}

\hypertarget{ux3b5ux3c0ux3b5ux3beux3b5ux3c1ux3b3ux3b1ux3c3ux3afux3b1}{%
\subsection{Επεξεργασία}\label{ux3b5ux3c0ux3b5ux3beux3b5ux3c1ux3b3ux3b1ux3c3ux3afux3b1}}

\begin{itemize}
\item
  \texttt{git~clone~https://github.com/\textasciitilde{}your-username\textasciitilde{}/kallipos.git}
\item
  \texttt{git~submodule~update~-\/-remote~-\/-init}
\item
  \texttt{chmod~+x~make-latex.sh}
\item
  \texttt{mkdir~latex}
\end{itemize}

Από το make-latex.sh βάζουμε σε σχόλιο το
\texttt{sed~-i~s+figure+Εικόνα+g} (αυτή η εντολή λειτουργεί μόνο σε Mac
συστήματα)

Αλλάζουμε στο αρχείο \texttt{figure.lua} τη γραμμή 12
\texttt{src~=~".."~..~src} :arrow\_right: \texttt{src~=~"."~..~src}. Ο
λόγος που το κάνουμε αυτό είναι επειδή θέλουμε το αρχείο να ψάχνει στο
current directory και όχι σε ένα directory πίσω.

\hypertarget{ux3bcux3b5ux3c4ux3b1ux3c4ux3c1ux3bfux3c0ux3ae-ux3b1ux3c0ux3cc-.tex-ux3c3ux3b5-pdf}{%
\subsection{Μετατροπή από .tex σε
pdf}\label{ux3bcux3b5ux3c4ux3b1ux3c4ux3c1ux3bfux3c0ux3ae-ux3b1ux3c0ux3cc-.tex-ux3c3ux3b5-pdf}}

\begin{enumerate}
\item
  Κάνουμε πρώτα ένα merge των αρχείων tex σε ένα αρχείο book.tex (δεν το
  σβήνουμε καθώς θα χρειαστεί σε επόμενο παραδοτέο)
\end{enumerate}

\begin{itemize}
\item
  \texttt{pandoc~-s~latex/*.tex~-o~book.tex}
\end{itemize}

\begin{enumerate}
\item
  Μετατροπή του book.tex σε book.pdf
\end{enumerate}

\begin{itemize}
\item
  pandoc -N --variable ``geometry=margin=1.2in'' --variable
  mainfont=``το δικο σας font'' --variable sansfont=``το δικο σας font''
  --variable monofont=``το δικο σας fontr'' --variable fontsize=12pt
  --variable version=2.0 tex/book.tex --pdf-engine=xelatex --toc -o
  book.pdf
\end{itemize}

Η παραπάνω εντολή θα βγάλει ένα warning ότι χρησιμοποιείτε λάθος
μετατροπέα (xelatex) παρόλο που χρησιμοποιείτε τον σωστό. Δεν υπάρχει
κανέναν πρόβλημα απλά αγνοήστε το.

Για τα φίλτρα της lua πρέπει να τα δηλώσουμε μέσα στα .txt αρχεία που
θέλουμε να πειράξουμε και να τα βάλουμε και στο make-latex.sh για να
γίνουν οι εισαγωγές μας στο pdf. -
\texttt{pandoc~-\/-lua-filter=\textasciitilde{}your-filter\textasciitilde{}.lua}

Βοηθητικό documentation:

\begin{itemize}
\item
  https://pandoc.org/MANUAL.html\#
\item
  https://pandoc.org/lua-filters.html
\item
  https://garrettgman.github.io/rmarkdown/authoring\_pandoc\_markdown.html
\end{itemize}

Αν σας βοήθησε το guide κάντε ένα upvote
\href{https://github.com/courses-ionio/help/discussions/1151}{εδώ} για
να το βρουν και άλλοι.

Made with :heart: by
\href{https://github.com/Second-Time-Is-The-Charm/}{Second Time Is The
Charm}

\end{document}
