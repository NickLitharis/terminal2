\hypertarget{netlify-cli}{%
\section{Netlify-CLI}\label{netlify-cli}}

\hypertarget{windows}{%
\subsection{Windows}\label{windows}}

Απαραίτητα: \href{https://nodejs.org/en/download/}{node.js},
\href{https://rubyinstaller.org/downloads/}{ruby + jekyll},
\href{https://cli.netlify.com/}{Netlify cli} \#\#\# node.js install

\begin{itemize}
\tightlist
\item
  Ακολουθούμε απλά το install του και για καλό και για κακό βάζουμε και
  την επιλογή να κάνει install και το Chocolatey (δε δοκιμάστηκε χωρίς
  αυτό)
\end{itemize}

\hypertarget{ruby-jekyll}{%
\subsubsection{ruby + jekyll}\label{ruby-jekyll}}

\begin{itemize}
\tightlist
\item
  Ανοίγουμε το installer
\end{itemize}

\begin{figure}
\centering
\includegraphics{https://user-images.githubusercontent.com/45509916/200146200-ee3e7b04-9d70-41c3-9783-35678d46a9c2.png}
\caption{image}
\end{figure}

Σε αυτή τη φάση κάνουμε και τις 3 επιλογές

\begin{figure}
\centering
\includegraphics{https://user-images.githubusercontent.com/45509916/200145993-08c1436b-ce9c-4684-a170-1c82474a449c.png}
\caption{image}
\end{figure}

\begin{itemize}
\tightlist
\item
  Μόλις τελειώσει με τις επιλογές μπορούμε άπλα να το κλείσουμε.
\item
  Κοιτάζουμε αν το έχουμε εγκαταστήσει σωστά με την εντολή
  \texttt{jekyll\ -v}
\item
  Στη συνέχεια χρησιμοποιούμε την εντολή \texttt{bundler\ install} για
  να μας δημιουργήσει όλα τα απαραίτητα gems που χρειαζόμαστε.
\item
  Μετά από αυτό κάλο είναι να γίνει ένα reboot (το συνιστά και το
  installer)
\end{itemize}

\hypertarget{netlify-cli-setup}{%
\subsubsection{Netlify cli setup}\label{netlify-cli-setup}}

\begin{itemize}
\tightlist
\item
  Αφού έχουμε τελειώσει με όλα τα install
\item
  χρησιμοποιούμε την εντολή \texttt{npm\ install\ netlify-cli\ -g} για
  την εγκατάσταση του Netlify cli
\item
  κάνουμε \texttt{netlify} για να δούμε ότι έχει γίνει η εγκατάσταση
  όπου θα μας εμφανίσει το παρακάτω πίνακα
\end{itemize}

\begin{figure}
\centering
\includegraphics{https://user-images.githubusercontent.com/45509916/200147017-2995fe4f-2c45-4e80-a03b-960d2ffddd0a.png}
\caption{image}
\end{figure}

\textbf{ΔΕΝ ΞΕΧΝΑΜΕ ΝΑ ΒΑΛΟΥΜΕ ΤΗ ΣΩΣΤΗ VERSION ΤΗΣ RUBY netlify env:set
RUBY\_VERSION 2.6.2.47}

\begin{enumerate}
\def\labelenumi{\arabic{enumi}.}
\tightlist
\item
  Χρησιμοποιούμε την εντολή \texttt{netlify\ login} όπου θα μας ανοίξει
  τη σελίδα του Netlify για να κάνουμε authorize το τερματικό μας, στη
  συνέχεια πατάμε \texttt{netlify\ init}
\item
  πατάμε την εντολή που μας λέει να συνδεθούμε σε ήδη υπάρχον site και
  στη συνέχεια θα πατήσουμε για το repository που είναι το site μας.
\item
  \texttt{netlify\ build} για να χτιστεί η σελίδα μας και
  \texttt{netlify\ deploy} για να γίνει deploy η σελίδα σε ένα draft
  link στο οποίο μπορείτε να τεστάρετε τις αλλαγές σας χωρίς να
  χρειάζεστε να κάνετε commit πολλά πέρα από αυτά που χρειάζεστε και να
  έχετε ένα πολύπλοκο history στο repository σας.
\end{enumerate}

\hypertarget{arch-linux}{%
\subsection{Arch Linux}\label{arch-linux}}

\begin{enumerate}
\def\labelenumi{\arabic{enumi}.}
\tightlist
\item
  Εγκαθιστούμε τα nodejs, ruby
\end{enumerate}

\begin{itemize}
\tightlist
\item
  \texttt{sudo\ pacman\ -S\ nodejs}
\item
  \texttt{sudo\ pacman\ -S\ ruby\ base-devel}
\end{itemize}

\begin{enumerate}
\def\labelenumi{\arabic{enumi}.}
\setcounter{enumi}{1}
\tightlist
\item
  Προσθέτουμε το ruby στο PATH (το ίδιο ισχύει για όλα τα shells)
\end{enumerate}

\begin{itemize}
\tightlist
\item
  \texttt{echo\ \textquotesingle{}\#\ Install\ Ruby\ Gems\ to\ \textasciitilde{}/gems\textquotesingle{}\ \textgreater{}\textgreater{}\ \textasciitilde{}/.bashrc}
  *
\item
  \texttt{echo\ \textquotesingle{}export\ GEM\_HOME="\$HOME/gems"\textquotesingle{}\ \textgreater{}\textgreater{}\ \textasciitilde{}/.bashrc}
\item
  \texttt{echo\ \textquotesingle{}export\ PATH="\$HOME/gems/bin:\$PATH"\textquotesingle{}\ \textgreater{}\textgreater{}\ \textasciitilde{}/.bashrc}
\item
  \texttt{source\ \textasciitilde{}/.bashrc}
\end{itemize}

\begin{enumerate}
\def\labelenumi{\arabic{enumi}.}
\setcounter{enumi}{2}
\tightlist
\item
  Εγκαθιστούμε το Jekyll και το Bundler
\end{enumerate}

\begin{itemize}
\tightlist
\item
  \texttt{gem\ install\ jekyll\ bundler}
\item
  \texttt{bundler\ install}
\end{itemize}

\begin{enumerate}
\def\labelenumi{\arabic{enumi}.}
\setcounter{enumi}{3}
\tightlist
\item
  Εγκαθιστούμε το Netlify
\end{enumerate}

\begin{itemize}
\tightlist
\item
  \texttt{npm\ install\ netlify-cli\ -g}
\end{itemize}

Ακολουθούμε τις οδηγίες που αναφέρονται για τα Windows
