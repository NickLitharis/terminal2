\hypertarget{ux3baux3b1ux3c4ux3b1ux3c3ux3baux3b5ux3c5ux3ae-ux3b2ux3b9ux3b2ux3bbux3afux3bfux3c5}{%
\section{Κατασκευή
Βιβλίου}\label{ux3baux3b1ux3c4ux3b1ux3c3ux3baux3b5ux3c5ux3ae-ux3b2ux3b9ux3b2ux3bbux3afux3bfux3c5}}

\hypertarget{ux3c0ux3c1ux3bfux3b1ux3c0ux3b1ux3b9ux3c4ux3bfux3cdux3bcux3b5ux3bdux3b1}{%
\subsection{Προαπαιτούμενα}\label{ux3c0ux3c1ux3bfux3b1ux3c0ux3b1ux3b9ux3c4ux3bfux3cdux3bcux3b5ux3bdux3b1}}

\begin{enumerate}
\def\labelenumi{\arabic{enumi}.}
\tightlist
\item
  \texttt{sudo\ pacman\ -S\ pandoc}
\item
  \texttt{sudo\ pacman\ -S\ texlive-most} (2GB install)
\item
  \texttt{sudo\ pip\ install\ pandoc-fignos}
\end{enumerate}

\hypertarget{ux3b5ux3b3ux3baux3b1ux3c4ux3acux3c3ux3c4ux3b1ux3c3ux3b7-fonts}{%
\subsubsection{Εγκατάσταση
fonts}\label{ux3b5ux3b3ux3baux3b1ux3c4ux3acux3c3ux3c4ux3b1ux3c3ux3b7-fonts}}

\begin{enumerate}
\def\labelenumi{\arabic{enumi}.}
\tightlist
\item
  Βρισκόμαστε στο \texttt{\textasciitilde{}/} directory
\item
  \texttt{ls\ -la}
\item
  Βλέπουμε αν υπάρχει ο φάκελος \texttt{.fonts} αν δεν υπάρχει τότε
  \texttt{mkdir\ .fonts}
\item
  κατεβάζουμε σε zip το font που θέλουμε από την ιστοσελίδα του
\item
  unzip font-που-κατεβάσαμε.zip -d \textasciitilde/.fonts
\end{enumerate}

optional: \texttt{yay\ community/ttf-meslo-nerd-font-powerlevel10k}

\hypertarget{ux3b5ux3c0ux3b5ux3beux3b5ux3c1ux3b3ux3b1ux3c3ux3afux3b1}{%
\subsection{Επεξεργασία}\label{ux3b5ux3c0ux3b5ux3beux3b5ux3c1ux3b3ux3b1ux3c3ux3afux3b1}}

\begin{itemize}
\item
  \texttt{git\ clone\ https://github.com/\textasciitilde{}your-username\textasciitilde{}/kallipos.git}
\item
  \texttt{git\ submodule\ update\ -\/-remote\ -\/-init}
\item
  \texttt{chmod\ +x\ make-latex.sh}
\item
  \texttt{mkdir\ latex}
\end{itemize}

Από το make-latex.sh βάζουμε σε σχόλιο το
\texttt{sed\ -i\ s+figure+Εικόνα+g} (αυτή η εντολή λειτουργεί μόνο σε
Mac συστήματα)

Αλλάζουμε στο αρχείο \texttt{figure.lua} τη γραμμή 12
\texttt{src\ =\ ".."\ ..\ src} :arrow\_right:
\texttt{src\ =\ "."\ ..\ src}. Ο λόγος που το κάνουμε αυτό είναι επειδή
θέλουμε το αρχείο να ψάχνει στο current directory και όχι σε ένα
directory πίσω.

\hypertarget{ux3bcux3b5ux3c4ux3b1ux3c4ux3c1ux3bfux3c0ux3ae-ux3b1ux3c0ux3cc-.tex-ux3c3ux3b5-pdf}{%
\subsection{Μετατροπή από .tex σε
pdf}\label{ux3bcux3b5ux3c4ux3b1ux3c4ux3c1ux3bfux3c0ux3ae-ux3b1ux3c0ux3cc-.tex-ux3c3ux3b5-pdf}}

\begin{enumerate}
\def\labelenumi{\arabic{enumi}.}
\tightlist
\item
  Κάνουμε πρώτα ένα merge των αρχείων tex σε ένα αρχείο book.tex (δεν το
  σβήνουμε καθώς θα χρειαστεί σε επόμενο παραδοτέο)
\end{enumerate}

\begin{itemize}
\tightlist
\item
  \texttt{pandoc\ -s\ latex/*.tex\ -o\ book.tex}
\end{itemize}

\begin{enumerate}
\def\labelenumi{\arabic{enumi}.}
\setcounter{enumi}{1}
\tightlist
\item
  Μετατροπή του book.tex σε book.pdf
\end{enumerate}

\begin{itemize}
\tightlist
\item
  pandoc -N --variable ``geometry=margin=1.2in'' --variable
  mainfont=``το δικο σας font'' --variable sansfont=``το δικο σας font''
  --variable monofont=``το δικο σας fontr'' --variable fontsize=12pt
  --variable version=2.0 tex/book.tex --pdf-engine=xelatex --toc -o
  book.pdf
\end{itemize}

Η παραπάνω εντολή θα βγάλει ένα warning ότι χρησιμοποιείτε λάθος
μετατροπέα (xelatex) παρόλο που χρησιμοποιείτε τον σωστό. Δεν υπάρχει
κανέναν πρόβλημα απλά αγνοήστε το.

Για τα φίλτρα της lua πρέπει να τα δηλώσουμε μέσα στα .txt αρχεία που
θέλουμε να πειράξουμε και να τα βάλουμε και στο make-latex.sh για να
γίνουν οι εισαγωγές μας στο pdf. -
\texttt{pandoc\ -\/-lua-filter=\textasciitilde{}your-filter\textasciitilde{}.lua}

Βοηθητικό documentation:

\begin{itemize}
\item
  https://pandoc.org/MANUAL.html\#
\item
  https://pandoc.org/lua-filters.html
\item
  https://garrettgman.github.io/rmarkdown/authoring\_pandoc\_markdown.html
\end{itemize}

Αν σας βοήθησε το guide κάντε ένα upvote
\href{https://github.com/courses-ionio/help/discussions/1151}{εδώ} για
να το βρουν και άλλοι.

Made with :heart: by
\href{https://github.com/Second-Time-Is-The-Charm/}{Second Time Is The
Charm}
